%%%
%%%  Arquivo fonte relato.tex - Vers{\~a}o 9.94
%%%  Prof. Paulo S. Motta Pires /LECA/UFRN
%%%              Pode set utilizado/modificado livremente.
%%% para gerar PDF usar : 
%%%
%%%                  pdflatex relato.tex
%%% 
%%% As figuras podem ser geradas com o XFIG (exportar em eps)
%%% Lembrar de transform{\'a}-las de eps para pdf usando 
%%%                  eps2pdf figura.eps
%%%
\documentclass[a4paper,11pt]{article}
\title{Sugest{\~o}es para a Prepara{\c c}{\~a}o de Relat{\'o}rios T{\'e}cnicos
        \footnote{Vers{\~a}o 9.94}}
\author{}
\date{}
\usepackage[latin1]{inputenc}
\usepackage{epsfig}
\usepackage{float}
\usepackage{ae}
\usepackage[brazil]{babel}
\usepackage{indentfirst}
\usepackage{url}
\makeindex
%%%
%%%
\begin{document}
\maketitle
\thispagestyle{empty}
\vspace{5cm}
\begin{minipage}{\textwidth}
\flushright
{\large
Nome do Aluno \\
Disciplina, Per{\'\i}odo Letivo\\
Nome do Professor da Disciplina\\}
\vspace{7cm}
\center
{\large
Laborat{\'o}rio de Engenharia de Computa{\c c}{\~a}o e Automa{\c c}{\~a}o\\
Universidade Federal do Rio Grande do Norte\\
Natal-RN, M{\^e}s e Ano da Publica{\c c}{\~a}o}
\end{minipage} 
\pagebreak 

%%% FINAL DA CAPA

\begin{abstract}
Com o objetivo de padronizar a apresenta{\c c}{\~a}o de trabalhos, sugerimos este modelo para a confec{\c c}{\~a}o de Relat{\'o}rios T{\'e}cnicos. Os coment{\'a}rios, ao longo do texto, est{\~a}o escritos em {\it it{\'a}lico}.

 A vers{\~a}o mais recente deste documento est{\'a} dispon{\'\i}vel, nos formatos {\tt ps} e {\tt pdf}, em \url{http://www.leca.ufrn.br/~pmotta}.Sugest{\~o}es podem ser encaminhadas para \url{pmotta@leca.ufrn.br}.
\end{abstract}
\pagebreak

%%% FINAL DO RESUMO


\tableofcontents
\pagebreak

%%% FINAL DO SUM{\'A}RIO

\section{Introdu{\c c}{\~a}o}
Neste t{\'o}pico descrever, de forma suscinta, o contexto no qual o trabalho a ser apresentado est{\'a} inserido. Em seguida, descrever, de forma resumida, as partes constituintes do trabalho.\\
Considera{\c c}{\~o}es importantes :
\begin{enumerate}
\item {\it Utilizar, preferencialmente, um editor ou um  processador de texto} 
\item {\it Utilizar apenas uma das faces do papel}
\item {\it Utilizar folhas brancas}
\end{enumerate}
\pagebreak

%%% FINAL DA INTRODU{\c C}{\~A}O

\section{T{\'o}pico : T{\'\i}tulo do T{\'o}pico}
Desenvolver o assunto relativo ao t{\'o}pico
\begin{enumerate}
\item {\it A bibliografia utilizada deve ser referenciada no texto atrav{\'e}s de numera{\c c}{\~a}o sequencial, em algarismos ar{\'a}bicos, colocados entre colchetes. Por exemplo : } ``\ldots como descrito em \cite{b1} \ldots'' ou ``\ldots de acordo com o circuito el{\'e}trico apresentado na Figura 1-06 de \cite{b2}' \ldots''
\item {\it A lista das refer{\^e}ncias efetivamente consultadas, colocadas na se{\c c}{\~a}o} Refer{\^e}ncias, {\it deve seguir a ordem de chamada utilizada no texto.}
\end{enumerate} 

As equa{\c c}{\~o}es, centralizadas, devem ser referenciadas no texto atrav{\'e}s de numera{\c c}{\~a}o sequencial. Por exemplo, ``\ldots o sistema de equa{\c c}{\~o}es homog{\^e}neas representado pelas equa{\c c}{\~o}es anteriores, ter{\'a} solu{\c c}{\~a}o n{\~a}o-trivial se :\\
\begin{equation} \left |
\begin{array}{cccc}
J_{\nu}(ua) & 0 & -K_{\nu}(wa) & 0 \\
0 & J_{\nu}(ua) & 0 & -K_{\nu}(wa) \\
\frac{i \nu \beta}{u^2 a}J_{\nu}(ua) & -\frac{\omega \nu}{u} J'_{\nu}(ua) & \frac{i \nu \beta}{w^2 a}K_{\mu}(wa) & -\frac{\omega \nu}{w} K'_{\nu}(wa) \\[6pt]
\frac{\omega \epsilon_{1}}{u}J'_{\nu}(ua)& \frac{i \nu \beta}{u^2 a}J_{\nu}(ua) & \frac{\omega-\epsilon_{2}}{w}K'_{\nu}(wa) & \frac{i \nu \beta}{\omega^2 a}K_{\nu}(wa)
\end{array}
\right | = 0
\end{equation}
onde
\begin{equation}
f=\frac{J'_{\nu}(U)}{UJ_{\nu}(U)} \qquad\mbox{;}\qquad g=\frac{K'_{\nu}(W)}{WK_{\nu}(W)}
\end{equation}
sendo  $J_{\nu}(x)$ e $K_{\nu}(y)$ a fun{\c c}{\~a}o de Bessel e a fun{\c c}{\~a}o de Bessel modificada, respectivamente. Os argumentos dessas fun{\c c}{\~o}es s{\~a}o :
\begin{eqnarray}
U^2 &=& (ua)^2 = (k_{1}a)^2 - (\beta a)^2\\ 
W^2 &=& (wa)^2 = (\beta a)^2 - (k_{2}a)^2 
\end{eqnarray}
com
\begin{eqnarray}
\epsilon &=& \frac{\epsilon_{1}}{\epsilon_{2}} = \frac{n_{1}^2}{n_{2}^2}\\
 k_{i} &=& \omega^2 \mu \epsilon_{i} = k_{0}^2 n_{i} \qquad \mbox{;}\qquad  \mbox{i = 1,2}
\end{eqnarray}
e
\begin{eqnarray}
J'_{\nu}(U) &=& \frac{1}{2}[J_{\nu-1}(U) - J_{\nu+1}(U)]\\
K'_{\nu}(W) &=& -\frac{1}{2}[K_{\nu-1}(W) + K_{\nu+1}(W)] 
\end{eqnarray}

A linha acima de cada fun{\c c}{\~a}o representa a derivada da fun{\c c}{\~a}o com rela{\c c}{\~a}o ao seu argumento.{\'E} importante ressaltar que $\epsilon_{1}$ e $\epsilon_{2}$ s{\~a}o, respectivamente, a permissividade el{\'e}trica do n{\'u}cleo e da casca da fibra enquanto $\beta$ {\'e} a constante de propaga{\c c}{\~a}o dos modos dentro da estrutura.''

As Tabelas, tamb{\'e}m, devem ser referenciadas no texto atrav{\'e}s de numera{\c c}{\~a}o sequencial. Por exemplo, ``\ldots{}conforme dados apresentados na Tabela 1{}\ldots''
\begin{table}[H]
  \centering
  \begin{tabular}{|r|r|r|r|r|r|}
    \hline
        Time &    Jogos &       PG &       PP &       GP &       GC \\
    \hline\hline
    Cruzeiro &        3 &        9 &        0 &       17 &        0 \\
        {\'I}bis &        3 &        3 &        6 &        8 &        9 \\
    Flamengo &        3 &        0 &        9 &        0 &       16 \\
    \hline
  \end{tabular}
  \caption{Classifica{\c c}{\~a}o ap{\'o}s tr{\^e}s rodadas}
\end{table}

As Figuras devem ser referenciadas no texto atrav{\'e}s de numera{\c c}{\~a}o sequencial. Por exemplo, ``\ldots{}aplica{\c c}{\~a}o cliente-servidor mostrada na Figura 1{}\ldots''
\begin{figure}[H]
        \centering
        %\includegraphics[height=6.0cm]{figura1.pdf}
        \caption{Aplica{\c c}{\~a}o Cliente-Servidor}
\end{figure}
\section{T{\'o}pico : T{\'\i}tulo do T{\'o}pico}
Desenvolver o assunto relativo ao t{\'o}pico
\section{Conclus{\~o}es}
{\it Descrever, resumidamente, o trabalho realizado. Quando cab{\'\i}vel, citar poss{\'\i}veis extens{\~o}es}
\pagebreak

%%% AP{\^E}NDICE

\appendix
\section{Exemplo de um Ap{\^e}ndice}
Desenvolvemos um programa de computador, na linguagem C, cuja listagem {\'e} apresentada na Figura 2.

\begin{figure}[H]
 \begin{verbatim}
     1  #include <stdio.h>
       
     2  main(argc, argv, envp)
     3  int     argc;
     4  char    *argv[];
     5  char    *envp[];        
       
     6  {
     7          int i;
     8          for (i = 0; envp[i] != (char *) 0; i++)
     9                  printf("%s/n", envp[i]);
    10          
    11          exit(0);
    12  }
\end{verbatim} 
\caption{Exemplo de um programa}
\end{figure}
\vspace{2.5cm}
{\it Em geral, os Ap{\^e}ndices s{\~a}o utilizados para colocar listagens de programas de computador, exemplos mais completos de utiliza{\c c}{\~a}o de programas, apresenta{\c c}{\~a}o mais elaborada de algum t{\'o}pico abordado no corpo do programa (dedu{\c c}{\~a}o de alguma equa{\c c}{\~a}o, por exemplo), desenvolvimento de algum t{\'o}pico perif{\'e}rico objetivando complementar alguma informa{\c c}{\~a}o.}
\pagebreak

%%% BIBLIOGRAFIA

\begin{thebibliography}{99}
\bibitem{b1} T.~0etiker, The Not So Short Introduction to \LaTeXe{}, obtido em \url{ftp://ftp.dante.de/tex-archive/info/lshort}
\bibitem{b2} F.~Monssen, PSpice with Circuit Analysis, Merril, pp.22, 1993
\end{thebibliography}

\end{document}


